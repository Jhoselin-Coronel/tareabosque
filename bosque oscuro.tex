\documentclass[12pt a4paper]{article}
\usepackage[utf8]{inputenc}
\usepackage{lipsum}
\title{La hipotesis del bosque oscuro}
\author{Jhoselin Coronel}
\date{02 de abril de 2024}
\begin{document}
\maketitle
\section{Introducción}
Han analizado en vivo y en directo las momias extraterrestres y han dicho que no están juntas pegadas esa es la prueba irrefutable definitiva ahora que dices pues nuevamente les digo que no tienen idea de como funciona la ciencia a estas momias las han analizado expertos de todo el mundo desde el 2017 en la gran mayoría han dicho que son falsas que están armadas que ahora salgan otros expertos a decir que es verdadera no quiere decir que sean verdaderas así no funciona la ciencia asi no se demuestran las cosas no se demuestran observando imagen en vivo o videos por dos horas, la evidencia científica que se ha mostrado durante todos estos años  sobre estas momias alienígenas y que hacen decir que son falsas es mucho mayor que la evidencia que muestra los que las defienden no es que no se hayan estudiando científicamente como dice el señor si se han estudiando. Esta conclusión ha sido respaldada por el paleontólogo Rodolfo Salas Gismondi, el antropólogo Alain Froment, la bioarqueologa Elsa Tomasto, el veterinario forense Jhon Islachin, el paleontólogo Alexey Bondarev, el paleontólogo Julien Benoit, el antropólogo forense Flavio Estrada. 
Es innegable que nos fascina la idea de los extraterrestres como acabamos de ver la gran mayoría quisiéramos ver uno ya , aunque sea en tamales pero porque nos es tan fácil encontrar inteligencia porque no hemos recibido ni una señal de ellos.

\section{La paradoja de fermi}
Que dice que se tantas estrellas y planetas en el cosmos donde esta el resto de vida inteligente porque no se manifiesta. La primera es que no existe ninguna paradoja estamos verdaderamente solos, a veces las explicaciones mas simples son las mas posibles, en realidad somos los primeros en desarrollar vida inteligente y no hay nadie mas que pueda decirnos hola, otra es que los extraterrestres están evitando intencionalmente el contacto con nosotros quizás estos seres tengan una política similar a la directiva principal de star trek.
\section{El gran filtro}
Es que las sociedades extraterrestres podrían estar millones de años por delante d nosotros haciendo que nuestras señales de radio sean tan anticuadas como los teléfonos de 50 años para nosotros y las transmisiones de radio o están muy avanzados como para detectar nuestra tecnología que a su ritmo es muy anticuada otra posible solución es que tienen una vida útil limitada y no pueden establecer una comunicación a largo plazo. Sugiere que ninguna civilización avanzada sobrevive el tiempo suficiente para existir junto a civilizaciones vecinas prosperas, motivos para destruir una civilización hay un montón puede haber una guerra nuclear un cambio climático que vuelve al planeta como venus la pandemia super poderosa estar cerca una supernova ser constantemente bombardeado con asteroides el universo es un lugar muy hostil mas hostil que el corazón 
\section{La hipotesis del bosque oscuro}
El comienza con una conversación entre un joven profesor se sociología y ex astrónomo Luo Ji y la madre de un amigo fallecido Ye Wenjie, mientras visita la tumba de la hija de Ye luego se sorprende cuando llegue le sugiere que cree un nuevo campo llamado sociología cósmica según Ye el centro de este campo se encuentra una suposición clave en la ecuación de Drake el dice imaginemos una gran cantidad de civilizaciones distribuidas por todo el  universos del orden del numero de estrellas observables montones y montones de hechos esas civilizaciones constituyen el cuerpo de  una sociedad cósmica.
\section{La sociologia cosmica}
Estudio de la cosmosociología son las leyes del universo y los patrones de orden cósmico que se repiten tanto en la vida de los seres humanos como en la naturaleza en general. La cosmosociología procura comprender el funcionamiento del universo, desde sus escalas más pequeñas (como las partículas subatómicas) hasta las más grandes (como los sistemas estelares), decodificando el orden y patrones comunes en todas partes.
Considera que el ser humano no solo está en el universo, sino que primordialmente es parte del universo y, por ende, las mismas leyes que hacen funcionar la naturaleza se aplican también a los hechos cotidianos de cada persona. Plantea que, al ser el universo el sistema más eficiente y de óptimo funcionamiento, conocer sus patrones y leyes de organización y aplicarlos conscientemente a nuestra propia organización social es la forma más eficiente de evolucionar y lograr nuestros propósitos en común.
\section{Cadena de sospechas}
Las civilizaciones maliciosas tienen una inclinación a atacar a otras debido a su naturaleza o su deseo de obtener territorio, recursos entre otros motivos .
Las civilizaciones benevolentes no suelen atacar a otras a menos que se sientan amenazas sin embargo cuando se encuentran con otras civilizaciones con las que es extremadamente difícil comunicarse y cuyas intenciones son inciertas pueden considerar la opción de tomar medias preventivas.
\section{Mensaje}
El universos es un bosque oscuro cada civilización es común cazador armado que acecha en medio de los arboles como un espectro apartando con cuidado las ramas que obstruyen su camino y tratando de moverse en silencio incluso la respiración se realiza con preocupación el cazador debe ser cauteloso ya que en el bosque hay otros cazadores sigilosos como el si se encuentra con otra forma de vida ya sea otro cazador un ser angelical o demoniaco un ser vulnerable o un anciano o un ser fantástico o un semidios solo hay una opción a abrir fuego y eliminarlo en este bosque con un peligro constante ya que cualquier forma de vida de revele su existencia será rápidamente aniquilada.
\section{Conclusion}
La hipótesis del bosque oscuro supone que las civilizaciones alienígenas, que son muchas en todo el universo, son silenciosas y paranoicas. Por ello, considerarían a otros tipos de vida inteligente como una amenaza, y destruiría cualquier vida naciente que haga notar su presencia. Esta teoría es explicada detalladamente en la novela de El Bosque Oscuro, de Liu Cixin. Incluso el autor indica cómo interactuar con vida extraterrestre potencialmente hostil. Para el escritor, las otras formas de vida nos consideran una amenaza y están dispuestas a todo con tal de mantenerse a salvo. Por ello consideran peligroso el contacto con humano, dado que podríamos intentar eliminar a quien revelara su ubicación. Esta hipótesis sería el motivo por el que no se hayan descubierto señales de radio extraterrestres, a pesar de disponer de las herramientas para captarlas, y de que hemos enviado accidentalmente nuestras señales de radio al espacio.
\end{document}